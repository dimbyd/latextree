% !TEX root = main.tex

\section{Videos}
Embedding videos in PDFs is difficult and unreliable, it's much easier to include hyperlinks to online videos using the \texttt{url} and \texttt{href} commands.

\subsection{\texttt{url} commands}
Please watch this video: \url{https://www.youtube.com/watch?v=oCDXhvXye9E}.

\subsection{\texttt{href} commands}
Please watch this \href{https://www.youtube.com/watch?v=oCDXhvXye9E}{video}.

\subsection{Floating \texttt{video} environments}
We define a new float called {\tt video} which uses the custom {\tt includevideo} command. This command is intended to mirror the way that {\tt includegraphics} is used within {\tt figure} floats. In the PDF version this is again rendered as a hyperlink, but {\tt LatexTree} embeds the video into the webpage.

Why does it need to be a {\tt float} instead of a simple environment?
\begin{itemize}
\item For PDF it's simply a centred/displayed URL in Latex (one line).
\item For HTML there's no need to float it!
\end{itemize}
Nevertheless it gives displayed videos the look and feel of figures and tables in terms of boxes, captions, etc. and we can have a list-of-videos ({\tt lov}) too.

To get an embed code for a YouTube video, click on {\tt Share->Embed} under the video and it will generate an entire {\tt iframe} tag from which the URL can be easily recovered.

\begin{video}
\centering
\includevideo[scale=0.5]{https://www.youtube.com/embed/oCDXhvXye9E}
\caption{Listen and learn folks!\label{vid:orange}}
\end{video}

\subsection{Beltram Vol. I}
\begin{video}
\centering
\small
\begin{tabular}{|c||c|}
\hline
\includevideo{https://www.youtube.com/embed/TZI0bwuF_bU}
&
\includevideo{https://www.youtube.com/embed/DprIo82yALY}
\\
Acid Over & We Gonna Rock \\ 
\hline\hline
\includevideo{https://www.youtube.com/embed/6iDz648nzT8}
&
\includevideo{https://www.youtube.com/embed/2MtUXW4_0BY}
\\
Passion & Last Rhythm \\
\hline
\end{tabular}
\caption{Four aces $\heartsuit\diamondsuit\clubsuit\spadesuit$\label{vids:choons}}
\end{video}


