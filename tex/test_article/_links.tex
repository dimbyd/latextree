% !TEX root = main.tex

\section{Cross-references and citations}
\label{sec:links}

Here is a labelled theorem containing a labelled equation.
\begin{theo}[Pythagoras' Theorem]\label{thm:pythagoras}
\begin{equation}\label{eq:pythagoras}
a^2 + b^2 = c^2.
\end{equation}
\end{theo}

% Labels are attached to the parent container
% A second label will overwrite the first.
\subsection{Cross-references}
\label{ss:xrefs}
\begin{itemize}
\item Here is a ref -\ref{sec:intro}- to the introduction.
\item Here is a ref -\ref{sec:links}- to the current section.
\item Here is a ref -\ref{ss:xrefs}- to the current subsection.
\item Here is a ref -\ref{thm:pythagoras}- to the above theorem.
\item Here is an eqref -\eqref{eq:pythagoras}- to the equation in the above theorem.
\end{itemize}

\subsection{Citations}
\begin{itemize}
\item A citation -\cite{grimmett01}- to the first bibtex entry.
\item A citation -\cite{hogg05}- to the second bibtex entry.
\item A citation -\cite{grimmett01,hogg05}- to the first two bibtex entries.
\end{itemize}

\subsection{\tt hyperref}
\begin{itemize}
\item Here is a url: -\url{http://www.bbc.co.uk/}-.
\item Here is some -\href{http://www.bbc.co.uk/}{hyperlinked text}-.
\item Here is a -\hyperref[sec:intro]{named cross-reference}- to the introduction.
\item here is an autoref -\autoref{sec:intro}- to the introduction.
\item Here is a nameref -\nameref{sec:intro}- to the introduction.
\end{itemize}

\subsection*{Document-level labels}
For a viewable links to the document-level label we need to use the \texttt{hyperref} command as follows: -\hyperref[doc:testdoc]{back-to-top}-. There is no number or title associated with the document-level so other xref commands don't know what to display. Document-level labels might be useful to organise documents within and across modules.
